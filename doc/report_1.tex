\documentclass[sigconf, authorversion]{acmart}

\usepackage{svg}
\usepackage{hyperref}
\usepackage[]{siunitx}
\PassOptionsToPackage{obeyspaces}{url}
\usepackage{dblfloatfix}

\AtBeginDocument{%
  \providecommand\BibTeX{{%
    \normalfont B\kern-0.5em{\scshape i\kern-0.25em b}\kern-0.8em\TeX}}}

\urlstyle{tt}

\settopmatter{printacmref=false}

\graphicspath{{../img/}}

\usepackage{currfile-abspath}

\sisetup{group-separator = {\,}}

\getmainfile % get real main file (can be different than jobname in some cases)
\getabspath{\themainfile} % or use \jobname.tex instead (not as safe)
\let\mainabsdir\theabsdir % save result away (macro will be overwritten by the next \getabspath
\let\mainabspath\theabspath % save result away (macro will be overwritten by the next \getabspath

\begin{document}

\title{Information Retrieval System for European Union Legislation}

\author{Diogo Miguel F. Rodrigues}
\email{up201806429@edu.fe.up.pt}
\affiliation{%
  \department{M.EIC}
  \institution{Faculty of Engineering of the University of Porto}
  \city{Porto}
  \country{Portugal}
}

\author{João António C. V. B. Sousa}
\email{up201806613@edu.fe.up.pt}
\affiliation{%
  \department{M.EIC}
  \institution{Faculty of Engineering of the University of Porto}
  \city{Porto}
  \country{Portugal}
}

\author{Rafael Soares Ribeiro}
\email{up201806330@edu.fe.up.pt}
\affiliation{%
  \department{M.EIC}
  \institution{Faculty of Engineering of the University of Porto}
  \city{Porto}
  \country{Portugal}
}

\renewcommand{\shortauthors}{Rodrigues, Sousa and Ribeiro}

\begin{abstract}
    EUR-Lex is an European legislation database that offers access to European Union (EU) law,
    case-law by the Court of Justice of the EU and other public EU documents.
    This project aims to retrieve, process and prepare the data from this database, in order to create an information retrieval system of EU legislation.
    Various methods were used to achieve this, from the data collection and filtering phase, to the exploration of the useful data and finally its application in a search system.
\end{abstract}

%%
%% The code below is generated by the tool at http://dl.acm.org/ccs.cfm.
%% Please copy and paste the code instead of the example below.
%%
\begin{CCSXML}
<ccs2012>
<concept>
<concept_id>10002951.10003317</concept_id>
<concept_desc>Information systems~Information retrieval</concept_desc>
<concept_significance>500</concept_significance>
</concept>
<concept>
<concept_id>10002951.10003317.10003371.10003381.10003382</concept_id>
<concept_desc>Information systems~Structured text search</concept_desc>
<concept_significance>500</concept_significance>
</concept>
</ccs2012>
\end{CCSXML}

\ccsdesc[500]{Information systems~Information retrieval}
\ccsdesc[500]{Information systems~Structured text search}

%%
%% Keywords. The author(s) should pick words that accurately describe
%% the work being presented. Separate the keywords with commas.
\keywords{datasets, information processing, information retrieval, full-text search}

%% A "teaser" image appears between the author and affiliation
%% information and the body of the document, and typically spans the
%% page.

\maketitle

\section{Introduction}
\label{intro}
The European Union (EU) is an economic and political union of 27 member states of Europe. Its predecessors were the European Community for Coal and Steel (ECCS) and the European Economic Community (ECC), created in 1951 and 1958 in the wake of the Second World War. The EU maintains the fundamental goals the ECCS and ECC had of promoting peace and economic cooperation among countries so that a new war among European nations would be economically infeasible. The EU has over time expanded its objectives so as to protect what has been known as the \textit{European Values}, encoded into the Charter of Fundamental Rights \cite{cfr} and the founding treaties of the EU \cite{teu, tfeu}.
The Commission has the initiative for legislation, citizens are entitled to participate in the shaping of the EU legislation through the election of the European Parliament, and their respective state governments through the European Council, being the European Court of Justice responsible for enforcing EU law.

EU law is a system of rules operating within the EU member states, and covers topics from Constitutional and Administrative Law to Freedom of Movement, Trade and Public Regulation, Natural Resource Management and Social Market Regulations.
The three main European legislation databases are EUR-Lex, PreLex and OEIL. EUR-Lex contains information of all the documents of the Official Journal (OJ) of the EU starting at the year of 1951, PreLex has complete records starting at 1974 and is an extension of EUR-Lex that has many of the same documents, but focuses on inter-institutional procedures and data concerning different stages in the decision-making process and law that has not been settled yet, meaning it is most likely to contain drafts of EU legislation. Lastly, OEIL is the database of the European Parliament and stores inter-institutional decision-making in the EU but from the point of view of the Parliament.

\section{Aspects of European legislation}

\begin{figure*}[b]
  \includegraphics[width=\textwidth]{diagram-pipeline.drawio}
  \caption{Data processing pipeline.}
\end{figure*}

\subsection{Identifier systems}

Most documents in EUR-Lex are assigned a unique, language independent CELEX (\textit{Communitatis Europae Lex}) number \cite{celex}. It has the format: Sector number -- Year (4 digits) -- Doc. type -- Doc. number.
ECLI (European Case-Law Identifier) is a uniform identifier used primarily in judicial decisions \cite{ecli}.
ELI (European Legislation Identifier) is a URI that can be read and used by humans and computers to refer to legislation. It is mostly associated with the OJ, and is endorsed by the European Council \cite{eli}.

\subsection{Classification systems}

The institutions of the EU use three classification systems under which they classify documents: the Directory of European Union Legislation, EuroVoc and subject matter.

The Directory of European Union Legislation (or otherwise \textit{directory codes}) is a numeric classification system used primarily by EUR-Lex. It assigns each area/sub-area of interest a number, and each area may contain many sub-areas; a directory code is a dot-separated list of area numbers. For example, a document related to agricultural research has directory code \texttt{03.30.50}: \textit{Agriculture} (\texttt{03}) $\rightarrow$ \textit{Agricultural structures} (\texttt{03.30}) $\rightarrow$ \textit{Agricultural research} (\texttt{03.30.50}) \cite{directory-legal-acts}.

The EuroVoc is a multilingual, multidisciplinary thesaurus covering the activities of the EU. It contains terms in 23 EU languages and 3 languages of candidate EU members, and is used by European institutions, the Publications Office of the EU, national and regional parliaments in Europe, and governments and private users around the world \cite{eurovoc}.

The subject matter authority table (AT) is a controlled vocabulary containing the concepts used for the indexation of notices published on EUR-Lex. It differs from directory codes and the EuroVoc in that the AT is strictly aligned with the evolution of EU policies cited in the different treaties of the EU \cite{subject-matter-at}.

\section{Dataset preparation}

\subsection{Metadata collection} \label{ssec:metadata-collection}

The dataset selected for this project was the EUR-Lex database. By making use of the \url{api.epdb.eu} API (namely of the \url{api.epdb.eu/eurlex/document} endpoint), we extracted a list of \SI{138911}{} document metadata entries, which we stored in file \texttt{raw.json}, with a size of \SI{387.8}{\mega\byte}.

For each document, the following fields were obtained:

\begin{itemize}
    \item \texttt{form}: Form (e.g., Agreement, Recommendation, ...)
    \item \texttt{date\_document}: Document date; generally date of signature.
    \item \texttt{title}: Document title.
    \item \texttt{oj\_date}: Date of publication in the OJ.
    \item \texttt{of\_effect}: Date of effect.
    \item \texttt{end\_validity}: Date the document validity ends.
    \item \texttt{addressee}: Addressee of the document (e.g., countries to which the document concerns).
    \item \texttt{subject\_matter}: Subject matters the document concerns.
    \item \texttt{directory\_codes}: List of directory codes.
    \item \texttt{eurovoc\_descriptors}: List of EuroVoc terms.
    \item \texttt{legal\_basis}: List of EU legislation identifiers (mostly CELEX IDs) that forms the legal basis of the current document.
    \item \texttt{relationships}: List of relationships between this document and other EUR-Lex documents (amendments, legal basis, ...).
    
    \item \texttt{eurlex\_perma\_url}: Permanent link of the document in the EUR-Lex website.

    \item \texttt{text\_url}: URL where document content is available.

    \item \texttt{doc\_id}: Document ID in the API.
    \item \texttt{api\_url}: URL of the document in the API
    \item \texttt{prelex\_relation}: List of documents in the PreLex database related to the current document.
    \item \texttt{internal\_ref}: Internal reference number of the document in the responsible body's separate database.
    \item \texttt{additional\_info}: Mostly used when the document has complex rules for start/end of validity.
\end{itemize}

\subsection{Data cleaning}
\label{ssec:data-cleaning}

After obtaining the metadata, we parsed \texttt{eurlex\_perma\_url} to obtain the CELEX identifiers. We decided to use CELEX as the primary document identifier, due to its ubiquity and wide support by the EUR-Lex website.
The following variables were not of interest for this project and were removed: \texttt{text\_url} (later on we used the CELEX to query the EUR-Lex website), \texttt{doc\_id}, \texttt{api\_url}, \texttt{prelex\_relation}, \texttt{internal\_ref} and \texttt{additional\_info}. We also renamed \texttt{date\_document} to \texttt{date}.

Finally, we converted the metadata to CSV format, which is more useful than JSON in pipelines, because JSON deserialization requires the whole file to be read at once, while CSV can be read one line at a time, although it is not as flexible as JSON. We encoded list fields (\texttt{subject\_matter}, \texttt{directory\_codes} and \texttt{eurovoc\_descriptors}) by joining list items with semicolons, after concluding that none of those fields' items contained semicolons.
The \SI{138911}{} processed entries were stored in \texttt{processed.csv} (\SI{60.6}{\mega\byte})

After exploring the \texttt{addressee} field, we concluded that the \url{api.epdb.eu} API truncated the \texttt{addressee} field to 255 characters. We nevertheless decided to keep this field, as although it has incomplete information, it may still be valuable data.

\subsection{Data filtering}

The additional data filtering step was defined to allow the flexibility required to filter out some documents based on deterministic criteria, in case the information processing and retrieval tool struggled with more data than required for this project. It is currently just a pipe between its input and output, as we did not yet find any issues with the large amount of data in the data processing stage. This left us with a file \texttt{filtered.csv} that is identical to \texttt{processed.csv} in size and contents.

\subsection{Text collection}

Using the CELEX identifiers from \texttt{filtered.csv}, we constructed URLs with format \path{https://eur-lex.europa.eu/legal-content/EN/TXT/HTML/?uri=CELEX:<}CELEX identifier\texttt{>} to retrieve document contents in HTML format, with the HTML files mostly containing only the document contents, or otherwise some headers. HTTP requests were made to the EUR-Lex website using the Python3 library \texttt{urllib} \cite{urllib}, and the HTML files were converted to plain text by removing all HTML tags with library \texttt{BeaufitulSoup} \cite{beautifulsoup}.

For some documents we were not able to retrieve their contents using the abovementioned URLs. Most such documents did not have a version in English, or instead had a PDF version and no  HTML version. We decided not to handle those files because the vast majority of the \SI{138911}{} files from previous steps had an HTML version in English, and having to consider documents in other languages or parse PDF files would add another layer of complexity when we already had more than enough data for this project.

Each document's contents was stored in a separate \textit{.txt} file named \texttt{texts/<}CELEX identifier\texttt{>.txt}. A complete list of documents for which contents were retrieved is available at \texttt{texts.txt}, one CELEX identifier per line (this file is essentially equivalent to listing all files in directory \texttt{texts} without their extensions). Out of the \SI{138911}{} documents we knew of from previous data processing steps, we were able to find text for \SI{99903}{} ($71.92\%$). The file \texttt{texts.txt} has a size of \SI{1.2}{\mega\byte}, and the total size of the \texttt{texts/} folder containing all document texts we could obtain is \SI{1.55}{\giga\byte}.

\subsection{Data enrichment}

Finally, we processed the text files to remove newlines in cases where there were three or more sequential newlines, and combined the document metadata from \texttt{filtered.csv} with the document contents in \texttt{texts/}, keeping only the documents for which we were able to obtain contents. The result of this operation was stored in \texttt{combined.csv}, with \SI{99903}{} entries and a size of \SI{1.31}{\giga\byte}.

\section{Dataset fields}
\label{sec:dataset-fields}

After completing all steps of data preparation, we obtained a dataset \texttt{combined.csv} with \SI{99903}{} entries and a size of \SI{1.31}{\giga\byte}. It has the fields described in Table \ref{tab:fields}.
The class diagram for the dataset is presented in Figure \ref{fig:class-diagram}.
Dates are represented in the format \texttt{YYYY-MM-DD}, and each date field is zero-padded when needed.

\begin{table}[ht]
    \centering
    \caption{Fields of the final dataset.} \label{tab:fields}
    \begin{tabular}{@{}l|c|l@{}}
        \textbf{Field}               \textbf{Type}   & \textbf{Description}    \\ \hline
        \texttt{celex}                  & text            & CELEX identifier        \\
        \texttt{form}                   & category        & Form of document        \\
        \texttt{date}                   & date            & Date of signature       \\
        \texttt{title}                  & text            & Title                   \\
        \texttt{oj\_date}               & date            & Date of pub. in OJ      \\
        \texttt{of\_effect}             & date            & Date of effect          \\
        \texttt{end\_validity}          & date            & End of validity         \\
        \texttt{addressee}              & text            & Addressee(s)            \\
        \texttt{subject\_matter}        & list            & Subject matters         \\
        \texttt{directory\_codes}       & list            & Directory codes         \\
        \texttt{eurovoc\_descriptors}   & list            & EuroVoc descriptors     \\
        \texttt{legal\_basis}           & list            & IDs of legal basis      \\
        \texttt{relationships}          & list            & IDs of related docs.    \\
        \texttt{text}                   & text            & Content of document     \\
    \end{tabular}
\end{table}

\begin{figure}
  \includegraphics[width=\linewidth]{diagram-class-diagram.drawio}
  \caption{Class diagram for conceptual model.}
  \label{fig:class-diagram}
\end{figure}

\section{Data exploration}
\label{sec:data-exploration}

Data exploration was performed to assert information like common keywords, form types and distribution of document publication date. Data exploration was performed using a Jupyter Notebook, running chunks of Python3 code.

Missing data can be caused by a variety of factors, but it is mostly associated to human error and the evolving metadata needs, which means certain older documents have empty fields, probably because at the time the database did not have all the fields it currently does.

\begin{table}[ht]
    \centering
    \caption{Missing values per field.} \label{tab:missing}
    \begin{tabular}{@{}l|r|r@{}}
        \textbf{Field}              & \textbf{Missing}& \textbf{\%} \\ \hline
        \texttt{date}                   & 1 615            & 1.62                    \\
        \texttt{title}                  & 0               & 0.00                    \\
        \texttt{oj\_date}               & 1 971            & 1.97                    \\
        \texttt{of\_effect}             & 14 483           & 14.50                   \\
        \texttt{end\_validity}          & 17 398           & 17.41                   \\
        \texttt{addressee}              & 74 951           & 75.02                   \\
        \texttt{subject\_matter}        & 15 400           & 15.41                   \\
        \texttt{directory\_codes}       & 17 310           & 17.33                   \\
        \texttt{eurovoc\_descriptors}   & 20 386           & 20.41                   \\
        \texttt{legal\_basis}           & 6 567            & 6.57                    \\
        \texttt{relationships}          & 869             & 0.87                    \\
    \end{tabular}
\end{table}

\texttt{addressee} was found to be the column by far with the highest missing data percentage at 75.02\%, Table \ref{tab:missing}. This lack of data might be due to the fact that most documents are general European law, and as such are directed at all member states that are part of the EU, regardless of EU membership at the time of publishing.

Column \texttt{end\_validity} has 17.41\% missing elements, which may be due to the indefinite validity of a document. The list columns \texttt{subject\_matter}, \texttt{directory\_codes} and \texttt{eurovoc\_descriptors} range from 15.41\% to 20.41\% missing elements which may be due to the mere absence of list items (e.g., there are no EuroVoc terms associated to the document due to its nature), or due to human error when inputting records in the database, information loss or non-applicability of some fields in certain documents.\par

An interesting observation is that \texttt{relationships} is not present only in 0.87\% of the entries, revealing the strong connection between the vast majority of EU legislation: laws that replace, complement or build upon older ones.

\subsection{Dates}

Document dates range from 24 Sep 1949 to 4 Oct 2013. \texttt{oj\_date} and \texttt{of\_effect} distributions are very similar to \texttt{date}, increasing until 1981 (second EU enlargement). In the 1981-2001 period the number of documents per year grew slowly, until late 2000 when the Nice European Council decided to speed up accession negotiations with the 12 countries that would later join the EU in 2004, causing the number of documents signed in 2001 to increase by almost $60\%$ when compared with 2000. 
This number has stayed somewhat stable and even decreased since 2002. \texttt{end\_validity} values range from 7 Jan 1954 to 31 Dec 2058, with the exception of \SI{34833}{} documents that have as expiration date 1 Jan 2100, which we interpret as the document not expiring.

\begin{figure}[H]
  \includesvg[width=\linewidth]{date-histogram.svg}
  \caption{Date histogram.}
\end{figure}

\subsection{Lists}

The dataset has 249 unique subject matters, being ``common commercial policy'' the one with the highest frequency appearing \SI{11961}{} times, followed by ``External relations'' (\SI{10478}{}) and ``Agriculture'' (\SI{8943}{}).
The dataset uses \SI{5204}{} unique EuroVoc descriptors, with the most frequent being ``import''  (\SI{5871}{}), ``export refund'' (\SI{4785}{}) and ``originating product'' (\SI{3837}{}).
Each document is related to an average of 5.93 other documents. The documents most referred to are \texttt{32007R1234} (regulation on the common organisation of agricultural markets), \texttt{21994A0103(74)} (\textit{Agreement on the European Economic Area}) and \texttt{11957E113} (\textit{Treaty establishing the European Economic Community}).

\subsection{Text fields}

\begin{figure}[ht]
    \centering
    \begin{minipage}[t]{0.5\linewidth}
        \centering
        \includesvg[width=\linewidth]{title-length-histogram.svg}
    \end{minipage}%
    \begin{minipage}[t]{0.5\linewidth}
        \centering
        \includesvg[width=\linewidth]{text-length-histogram.svg}
    \end{minipage}
    \caption{Text fields length histograms.}
\end{figure}

Title length varies between 6 and \SI{1531}{} characters, with an average length of 216.3 characters. The title length approximately follows a log-normal distribution, since it is mostly left-skewed and its tail (to the right) tends to get smaller quite slowly.
Text length is between 300 and 6.3M characters, with an average length of 15288.3 characters. It presents a distribution similar to \texttt{title}.

\section{Retrieval tasks}
\label{sec:retrieval tasks}

Here we present some of the information needs we expect to meet at a later stage of this project.

\begin{itemize}
    % \item How many types of documents are there?
    % \item In what year were published the highest amount of documents?
    \item Amends to Decision 1999/468/EC
    \item Legislation related to Portugal
    \item Most recent consolidated legislation on Agriculture
    % \item What is/was the document that is/was/will be in force for the longest period of time?
    \item Competition-related documents in force
    \item Data protection regulation after 2010
    \item Legislation on import-export with Canada
    \item Fishing quotas in the North Atlantic Sea
\end{itemize}

\section{Conclusion}

The fundamental goals of this project were to search and select relevant datasets, perform data exploration analysis, assess the datasets' quality, characterize its fields and identify information needs for the system to be developed.

We obtained a large dataset with data as coherent and complete as possible for the subset of documents we chose to analyse.

\bibliographystyle{ACM-Reference-Format}
\bibliography{\mainabsdir/../bib/bibliography-file.bib}

\end{document}
\endinput
